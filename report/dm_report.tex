% do not change these two lines (this is a hard requirement
% there is one exception: you might replace oneside by twoside in case you deliver 
% the printed version in the accordant format
\documentclass[11pt,titlepage,oneside,openany]{article}
\usepackage{times}


\usepackage{graphicx}
\usepackage{latexsym}
\usepackage{amsmath}
\usepackage{amssymb}

\usepackage{ntheorem}

% \usepackage{paralist}
\usepackage{tabularx}

% this packaes are useful for nice algorithms
\usepackage{algorithm}
\usepackage{algorithmic}

% well, when your work is concerned with definitions, proposition and so on, we suggest this
% feel free to add Corrolary, Theorem or whatever you need
\newtheorem{definition}{Definition}
\newtheorem{proposition}{Proposition}


% its always useful to have some shortcuts (some are specific for algorithms
% if you do not like your formating you can change it here (instead of scanning through the whole text)
\renewcommand{\algorithmiccomment}[1]{\ensuremath{\rhd} \textit{#1}}
\def\MYCALL#1#2{{\small\textsc{#1}}(\textup{#2})}
\def\MYSET#1{\scshape{#1}}
\def\MYAND{\textbf{ and }}
\def\MYOR{\textbf{ or }}
\def\MYNOT{\textbf{ not }}
\def\MYTHROW{\textbf{ throw }}
\def\MYBREAK{\textbf{break }}
\def\MYEXCEPT#1{\scshape{#1}}
\def\MYTO{\textbf{ to }}
\def\MYNIL{\textsc{Nil}}
\def\MYUNKNOWN{ unknown }

% simple stuff (not all of this is used in this examples thesis
\def\INT{{\mathcal I}} % interpretation
\def\ONT{{\mathcal O}} % ontology
\def\SEM{{\mathcal S}} % alignment semantic
\def\ALI{{\mathcal A}} % alignment
\def\USE{{\mathcal U}} % set of unsatisfiable entities
\def\CON{{\mathcal C}} % conflict set
\def\DIA{\Delta} % diagnosis
% mups and mips
\def\MUP{{\mathcal M}} % ontology
\def\MIP{{\mathcal M}} % ontology
% distributed and local entities
\newcommand{\cc}[2]{\mathit{#1}\hspace{-1pt} \# \hspace{-1pt} \mathit{#2}}
\newcommand{\cx}[1]{\mathit{#1}}
% complex stuff
\def\MER#1#2#3#4{#1 \cup_{#3}^{#2} #4} % merged ontology
\def\MUPALL#1#2#3#4#5{\textit{MUPS}_{#1}\left(#2, #3, #4, #5\right)} % the set of all mups for some concept
\def\MIPALL#1#2{\textit{MIPS}_{#1}\left(#2\right)} % the set of all mips


\begin{document}

\pagenumbering{roman}

% Title page
\begin{titlepage}
	\vspace*{2cm}
	\begin{center}
		{\huge Dog Breed Identification\\}
		\vspace{2cm} 
		{\large Data Mining I Team Project Report\\}
		\vspace{1.7cm}
		{presented by\\ 
			\vspace{1.0cm} 
			\underline{Team 5} \\
			\vspace{0.5cm} 
			Kengo Arao \\
			Michael Hung \\
			Tuan Anh Pham Le \\
			Sascha Marton \\
			Kacper Zurad \\
		}
		\vspace{1cm} 
		{submitted to the\\
			Data and Web Science Group\\
			Prof.\ Dr.\ Stuckenschmidt\\
			University of Mannheim\\} \vspace{2cm}
		{October 2017}
	\end{center}
\end{titlepage}

\tableofcontents

\newpage

% okay, start new numbering ... here is where it really starts
\pagenumbering{arabic}

\section{Application Area and Goals}
\label{cha:intro}

% add papers/ articles/ etc. in references.bib
% cite like this:
\cite{gys07}

\section{Structure and Size of the Data Set}
\label{sec:struc}

\section{Preprocessing}
\label{sec:prep}

\section{Data Mining}
\label{sec:class}

\subsection{K-NN}
\subsection{Naive Bayes}
\subsection{SVM}

% ...

\subsection{InceptionV3}

\newpage

\bibliographystyle{plain}
\bibliography{references}

\end{document}
